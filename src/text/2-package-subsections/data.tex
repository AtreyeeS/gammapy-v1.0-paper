\subsection{gammapy.data}
\label{ssec:gammapy-data}
\todo{Cosimo Nigro}

As illustrated in Fig.~\ref{fig:data_flow}, the \verb|gammapy.data| sub-package
implements the lowest data level, providing the input interface to gamma-ray data.
The \gammapy data model follows the \gadf specifications (Sec.~\ref{sec:introduction}),
therefore gamma-ray data compliant with those can be directly read with the package. 
Version X.X of the \gadf specification is supported in this release version.
As both the \gadf data model and \gammapy were initially conceived for \iact
data analysis, input data are typically grouped in \texttt{Observations}, corresponding
to stable periods of data aquisitions (typically $20 - 30\,{\rm min}$).
A \texttt{DataStore} object, gathering a collection of observations, can be created
porviding ancillary files containing information about the telescope observation
mode and the content of the data unit of each file. The \texttt{DataStore} allows
for selecting a list of observations based on specific filters.
As an illustrative example, Fig.~\ref{fig*:minted:gp_data} shows
how to create a \texttt{datastore} and how to obtain the observations corresponding
to given identification numbers (IDs).

The so-called DL3 files represented by the \texttt{Observation} class consist of
two types of elements: a list of gamma-ray events with relevant physical quantities
for the successive analysis (estimated energy, direction and arrival times) that
is handled by the \texttt{EventList} class and an instrument response function (IRF),
providing the response of the system, typically factorised in independent components
(see the description in Sec.~\ref{ssec:gammapy-irf}).
The separate handling of event lists and IRFs additionally allows for data from
other gamma-ray instruments to be read. For example, to read \fermi data, the
user can read separately their event list (already compliant with the \gadf
specifications) and then find the appropriate IRF class representing the response
functions provided by \fermi, see Sec.~\ref{ssec:fermi}.

\begin{figure}

	\import{code-examples/generated/}{gp_data}

	\caption{Using gammapy.data to access DL3 level data with a DataStore}
	\label{fig*:minted:gp_data}
\end{figure}