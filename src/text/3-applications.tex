\section{Applications}
\label{sec:applications}

Gammapy can be used for a variety of science cases by different IACT experiments. (Refer to publications)
In this section, we show a non-exhaustive list of some typical analysis that be performed.

\subsection{Classical spectral analysis}
\label{ssec:1D-analysis}
\subsection{Classical spectral analysis}
\label{ssec:1D-analysis}

\begin{figure*}[t]
	\centering
	\includegraphics[width=1.\textwidth]{figures/cta_galactic_center.pdf}
	\caption{CTA Galactic Center example}
	\label{fig:cta_galactic_center}
\end{figure*}


\todo{Axel}
A classical spectral analysis using Reflected Regions. Showing flux points, reflected regions on the CTA galactic center data

\subsection{A 3D likelihood analysis}
\label{ssec:3D-analysis}
\subsection{A 3D likelihood analysis}
\label{ssec:3D-analysis}
\todo{Laura}

Simulation of overlapping sources, 3D fitting, npred map, TS maps


\todo{What to do with } Figure~\ref{fig:fermi_ts_map} ?
Ref:~\citep{Stewart2009}
\begin{figure*}[t]
        \centering
        \includegraphics[width=1.\textwidth]{figures/fermi_ts_map.pdf}
        \caption{Fermi-LAT TS map in two energy bands} \label{fig:fermi_ts_map}
\end{figure*}

\subsection{Temporal analysis}
\label{ssec:temporal-analysis}
\subsection{Temporal analysis}
\label{ssec:temporal-analysis}
\todo{Atreyee}

Lightcurve in two energy bands (600 GeV - 1.5 TeV, and 1.5 TeV to 20 TeV) for
PKS2155-304 flare seen by H.E.S.S. as in our notebooks.

\begin{figure*}[t]
	\centering
	\includegraphics[width=1.\textwidth]{figures/hess_lightcurve_pks.pdf}
	\caption{
		H.E.S.S. PKS~2155-304 flare in two energy bands}
	\label{fig:hess_lightcurve_pks} \end{figure*}
\todo{Atreyee}
lightcurves and temporal models using the HESS PKS2155

\subsection{Multi instrument analysis}
\label{ssec:multi-instrument-analysis}
\subsection{Multi instrument analysis}
\label{ssec:multi-instrument-analysis}

\begin{figure}[t]
	\centering
	\includegraphics[width=1.\textwidth]{figures/multi_instrument_analysis.pdf}
	\caption{A multi-instrument analysis of the Crab Nebula}
	\label{fig:multi_instrument_analysis} 
\end{figure*}

\todo{Cosimo Nigro}
\begin{figure}[t]
	\centering
	\includegraphics[width=1.\textwidth]{figures/multi_instrument_analysis.pdf}
	\caption{A multi-instrument analysis of the Crab Nebula}
	\label{fig:multi_instrument_analysis} 
\end{figure*}

\subsection{Fitting using physical models}
\label{ssec:naima-analysis}
\subsection{Fitting using physical models}
\label{ssec:naima-analysis}

Figure \ref{fig:multi_instrument_analysis}. Add explanation text.
\todo{Cosimo}

Figure \ref{fig:multi_instrument_analysis}. Add explanation text.

\subsection{Population studies}
\label{ssec:population-studies}
\todo{Quentin Remy}