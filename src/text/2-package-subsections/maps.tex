\subsection{gammapy.maps}
\label{ssec:gammapy-maps}
The \code{gammapy.maps} sub-package provides classes that represent data
structures associated with a set of coordinates or a region on a sphere. In
addition it allows to handle an arbitrary number of non-spatial data
dimensions, such as time or energy. It is organized around three types of
structures: geometries, sky-maps and map axes, which inherit from the base
classes \code{Geom}, \code{Map} and \code{MapAxis} respectively.

The geometry object defines the pixelization scheme and map boundaries. It also
provides methods to transform between sky and pixel coordinates. Maps consist
of a geometry instance together with a Numpy data array containing the corresponding
map values. Map axes contain a sequence of ordered values which define bins on
a given dimension, spatial or not. Map axes can have physical units attached to
them, as well as define non-linear spaced bins. All map classes support operations such
as arithmetic operations with unit support, up- and downsampling along
extra axes, interpolation, resampling of extra axes, interactive visualisation
in notebooks and interpolation onto different geometries.

The sub-package provides a uniform API for the FITS World Coordinate System
(WCS), the HEALPix pixelization and region-based data structure
(see Figure~\ref{ig*:minted:gp_maps}).

\begin{figure}
	\import{code-examples/generated/}{gp_maps}

	\caption{
        Using \code{gammapy.maps} to create a WCS, a HEALPix and a region
		based data structures. The initialisation parameters include
        consistently the positions of the center of the map, the pixel
        size, the extend of the map as well as the energy axis definition.
        The energy minimum and maximum values for the creation of the
        \code{MapAxis} object can be defined as strings also specifying the
        unit. Region definitions can be passed as strings following
        the DS9 region specifications \url{http://ds9.si.edu/doc/ref/region.html}.
        }
    \label{ig*:minted:gp_maps}
\end{figure}

% itemize because it helps me write, could also just be paragraphs...
\subsubsection{WCS Maps}
The FITS WCS pixelization supports a different number of projections to
represent celestial spherical coordinates in a regular rectangular grid.
\gammapy provides full support to data structures using this pixelization
scheme. For details see ~\cite{Calabretta2002}. This pixelisation
is typically used for smaller regions of interests, such as pointed
observations.


\subsubsection{HEALPix Maps}
This pixelization scheme ~\citep{Calabretta2002} provides a
subdivision of a sphere in which each pixel covers the same surface area as
every other pixel. As a consequence, however, pixel shapes are no longer
rectangular, or regular.
This pixelisation is typically used for all-sky data, such as data
from the \hawc or \fermi observatory. \gammapy natively supports
the multiscale definiton of the HEALPix pixelisation and thus
allows for easy up and downsampling of the data. In addition to
the all-sky map, \gammapy also supports a local HEALPix
pixelisation where the size of the map is constrained to a given
radius.
For local nighbourhood operations, such as convolution \gammapy relies
on projecting the HEALPix data to a local tangential WCS grid.

\subsubsection{Region Maps}
In this case, instead of a fine spatial grid
dividing a rectangular sky region, the spatial dimension is reduced to a single
bin with an arbitrary shape, describing a region in the sky with that same
shape. Typically they are is used together with a non-spatial dimension, for
example an energy axis, to represent how a quantity varies in that dimension
inside the corresponding region. The region is represented on the local
tangential projection see \cite{Astropy regions}.

Additionally, the \code{MapAxis} class provides a uniform API for axes representing
bins on any physical quantity, such as energy or angular offset. The special
case of time is covered by the dedicated \code{TimeMapAxis}, which allows time bins to
be non-contiguous, as it is often the case with observation time-stamps. The
generic class \code{LabelMapAxis} allows the creation of axes for non-numeric
entries.
