\subsection{gammapy.analysis}
\label{ssec:gammapy-analysis}
\todo{Quentin, Axel...}

The \code{gammapy.analysis} sub-module provides a  high-level interface for the most
common use cases identified in gamma-ray analyses. The classes and methods
included can be used in Python scripts, notebooks or as commands within IPython
sessions. The high-level user interface can also be used to automatise
workflows driven by parameters declared in a configuration file in YAML format.
This way a full analysis can be executed via a single command line taking the
configuration file as input.

The \code{Analysis} class has the responsibility of orchestrating of the workflow
defined in the configuration \code{AnalysisConfig} objects and triggering the execution of
the \code{AnalysisStep} classes that defines the common use cases identified. These
steps include the following: observations selection from the DataStore,  data
reduction, excess map computation, model fitting, flux-points estimation, and
light-curves production. The structure of the \code{Analysis} class allows to define
custom AnalysisStep, so ones could design and configure arbitrarily complex
wokflows. \todo{the latter is not true for v1.0...}

