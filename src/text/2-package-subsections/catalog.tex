\subsection{gammapy.catalog}
\label{ssec:gammapy-catalog}
Comprehensive source catalogs are increasingly being provided by many high
energy astrophysics experiments. The \code{gammapy.catalog} sub-packages
provides a convenient access to the most important \gammaray catalogs.
Catalogs are represented by the \code{SourceCatalog} object, which
contains the actual catalog as an Astropy \code{Table} object.
Objects in the catalog can be accesed by row index, name of the
object or any association or alias name listed in the catalog.

Sources are represented in \gammapy by the \code{SourceCatalogObject}
class, which has the responsibility to translate the information
contained in the catalog to other \gammapy objects. This includes
the spatial and spectral model of the source, flux points and
light curves (if available) for individual objects. This
module works independently from the rest of the package, and the required
catalogs are supplied in \code{GAMMAPY\_DATA} repository.
The overview of currenly supported catalogs, the corresponding
\gammapy classes and references are shown in Table~\ref{tab:catalogs}.
Newly released relevant catalogs will be added in future.

\begin{table*}
    \begin{center}
        \begin{tabular}{llll}
         \hline
         Class Name & Shortcut & Description & Reference\\
         \hline
         \code{SourceCatalog3FGL} & \code{"3fgl"} & 3\textsuperscript{rd} catalog of \fermi sources & \cite{3FGL} \\
         \code{SourceCatalog4FGL} & \code{"4fgl"} & 4\textsuperscript{th} catalog of \fermi  sources & \cite{4FGL} \\
         \code{SourceCatalog2FHL} & \code{"2fhl"} & 2\textsuperscript{nd} catalog high energy \fermi  sources & \cite{2FHL} \\
         \code{SourceCatalog3FHL} & \code{"3fhl"} & 3\textsuperscript{rd} catalog high energy \fermi  sources & \cite{3FHL} \\
         \code{SourceCatalog2HWC} & \code{"2hwc"} & 2\textsuperscript{nd} catalog of \hawc sources & \cite{2HWC} \\
         \code{SourceCatalog3HWC} & \code{"3hwc"} & 3\textsuperscript{rd} catalog of \hawc sources & \cite{3HWC} \\
         \code{SourceCatalogHGPS} & \code{"hgps"} & \hess Galactic Plane Survey catalog & \cite{HGPS} \\
         \code{SourceCatalogGammaCat} & \code{"gammacat"} & Open source data collection & \citep{gamma-cat} \\
         \hline
         \end{tabular}
    \end{center}
    \label{tab:catalogs}
    \caption{Overview of supported catalogs in \code{gammapy.catalog}.}
\end{table*}

\begin{figure}
	\import{code-examples/generated/}{gp_catalogs}
	\caption{Using gammapy.catalogs: Accessing underlying model, flux points and
		lightcurve from the Fermi-LAT 4FGL catalog for the blazar PKS 2155-304}
	\label{codeexample:data}
\end{figure}
