\subsection{gammapy.modeling}
\label{ssec:gammapy-modeling}
%
\code{gammapy.modeling} contains all the functionality related to modeling and fitting
data. This includes spectral, spatial and temporal model classes, as well as
the fit and parameter API.

\subsubsection{Models}
\label{sssec:models}
Source models in \gammapy are four dimensional models which support two
spatial dimensions $\ell, b$, and energy dimension $E$ as well as
the time dimension $t$. To simplify the the definition of the
models, \gammapy uses a factorised representation of the total source
model:

\begin{equation}
    f(l, b, E, t) = F(E)  \cdot G(l, b, E) \cdot  H(t, E)
\end{equation}

Where the spectral model $F$ implicitly carries the total flux
norm of the model. The spatial model $G$ optionally depends
on energy and allows to support e.g. energy dependent
morphology models. The temporal model component $H$
optionally also supports an energy variable to
allow for spectral variations of the model with time.

Following the factorisation The models objects are grouped
into the following categories:

\begin{itemize}
	\item \code{SpectralModel}: models to describe spectral shapes of sources
	\item \code{SpatialModel}: models to describe spatial shapes (morphologies) of sources
	\item \code{TemporalModel}: models to describe temporal flux evolution of sources, such as
	      light and phase curves
\end{itemize}

The models follow a naming scheme which contains the category as a suffix to
the class name.

The spectral models include a special class of normed models, which have a
dimension-less normalisation. These spectral models feature a norm parameter
instead of amplitude and are named using the NormSpectralModel suffix. They
must be used along with another spectral model, as a multiplicative correction
factor according to their spectral shape. They can be typically used for
adjusting template based models, or adding a EBL correction to some analytic
model. The analytic spatial models are all normalized such as they integrate to
unity over the sky but the template spatial models may not, so in that special
case they have to be combined with a \code{NormSpectralModel}.

The \code{SkyModel} object represents factorised model that combine the spectral,
spatial and temporal model components (by default the spatial and temporal components are
optional). SkyModel objects represents additive emission components, usually
sources or diffuse emission, although a single source can also be modeled by
multiple components. To handle list of multiple \code{SkyModel} objects, \gammapy
has a \code{Models} class.

The model gallery provides a visual overview of the available models in
\gammapy. Most of the analytic models  commonly used in \gammaray astronomy are
built-in. We also offer a wrapper to radiative models implemented in the Naima
package~\cite{naima}. The modeling framework can be easily extended with
user-defined models. For example agnpy models that describe leptonic radiative
processes in jetted Active Galactic Nuclei (AGN) can wrapped into
\gammapy~\citep[see section3.5 of ][]{2021arXiv211214573N} .

\begin{figure}
	\import{code-examples/generated/}{gp_models}
	\caption{Using \code{gammapy.modeling.models} to define a source model with a
    spectral, spatial and temporal component. For convenience the model
    parameters can be defined as strings with attached units. The spatial model
    takes an addtional \code{frame} parameter which allow users to define
    the coordinate frame of the position of the model.
    }
	\label{fig*:minted:gp_models}
\end{figure}

\subsubsection{Fit}
\label{sssec:fit}

The \code{Fit} class provides methods to fit i.e., optimise parameters and estimate
parameter errors and correlations. It interfaces with a Datasets object, which
in turn is connected to a Models object containing the model parameters in its
Parameters object. Models can be unique for a given dataset, or contribute to
multiple datasets and thus provide links, allowing e.g., to do a joint fit to
multiple IACT datasets, or to a joint IACT and \fermi dataset. Many
examples are given in the tutorials.

\todo{Add fit code example...?}

The Fit class provides a uniform interface to multiple fitting backends:
“minuit”~\citep{iminuit}, “scipy”,~\citep{2020SciPy-NMeth}, and
“sherpa”~\citep{sherpa-2005,sherpa-2011}. Note that, for now, covariance matrix
and errors are computed only for the fitting with MINUIT. However depending on
the problem other optimizers can better perform, so sometimes it can be useful
to run a pre-fit with alternative optimization methods. In future we plan to
extend the supported fitting backends, including for example MCMC solutions.
\footnote{a prototype is available in gammapy-recipes,
	\url{https://gammapy.github.io/gammapy-recipes/_build/html/notebooks/mcmc-sampling-emcee/mcmc_sampling.html}
}
