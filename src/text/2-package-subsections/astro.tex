\subsection{gammapy.astro}
\label{ssec:gammapy-astro}
\todo{Atreyee}

The \code{gammapy.astro} sub-package contains utility functions for studying physical
scenarios in high energy astrophysics. The \code{gammapy.astro.darkmatter} module
computes the so  called J-factors and the associated gamma-ray spectra expected
from annihilation of dark matter in different channels according to the recipe
described in \cite{2011JCAP...03..051C}.

In the \code{gammapy.astro.source} sub-module, dedicated classes exist for modeling
galactic \gammaray sources according to simplified physical models, eg: SNR evolution
models \citep{1950RSPSA.201..159T, 1999ApJS..120..299T}, evolution of PWN during the
free expansion phase \citep{2006ARA&A..44...17G} or computation
of physical parameters in a pulsar assumed to be a simple dipole.

In the \code{gammapy.astro.population} sub-module there are dedicated tools
for simulating synthetic populations based on physical models derived from
observational or theoretical considerations for different classes of Galactic
VHE gamma-ray emitters: PWNe, SNRs \cite{1998ApJ...504..761C},
pulsars \cite{2006ApJ...643..332F, 2006MNRAS.372..777L, 2004A&A...422..545Y}
and gamma-ray binaries.

While the present list of use cases is rather preliminary, this can be enriched
with time with by users and/or developers according to future needs.
