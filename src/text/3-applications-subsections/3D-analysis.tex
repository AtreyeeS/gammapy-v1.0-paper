\subsection{3D Analysis}
\label{ssec:3d-analysis}
The 1D analysis approach is a powerful tool to measure the spectrum of an
isolated source. However, more complicated situations require a more careful
treatment. In a field of view containing several overlapping sources, the 1D
approach cannot disentangle the contribution of each source to the total flux in
the selected region. Sources with extended or complex morphology can result in
the measured flux being underestimated, and heavily dependent on the choice of
extraction region. Additionally, the 1D approach neglects the energy-dependence
of the PSF.

For such situations a more complex approach is needed, the so-called 3D
analysis. The three relevant dimensions are the two spatial angular coordinates
and an energy axis. In this framework, a combined spatial and spectral model
(that is, a \code{SkyModel}, see Section~\ref{ssec:gammapy-modeling}) is fitted to the
sky-maps that were previously derived from the data and bundled into a
\code{MapDataset} (see Sections~\ref{ssec:gammapy-makers} and~\ref{ssec:gammapy-datasets}).

A thorough description of the 3D analysis approach and multiple examples that
use \gammapy can be found in~\cite{Mohrmann2019}. Here we present a short
example to highlight some of its advantages.

Starting from the \irfs corresponding to the same three simulated \cta
observations used in Section~\ref{ssec:1d-analysis}, we can create a \code{MapDataset}
via the \code{MapDatasetMaker}. However, we will not use the simulated event lists
provided by \cta but instead, use the method MapDataset.fake() to simulate
measured counts from the combination of several SkyModel instances. In this
way, a DL4 dataset can directly be simulated. In particular we simulate:

\begin{enumerate}
    \item A point source located at (l=0\textdegree, b=0\textdegree) with a power-law
	      spectral shape.
    \item An extended source with Gaussian morphology located at (l=0.4\textdegree,
	      b=0.15\textdegree) with $\sigma$=0.2\textdegree and a log-parabola spectral
	      shape.
    \item A large shell-like structure centered around (l=0.06\textdegree,
	      b=0.6\textdegree) with a radius and width of 0.6\textdegree and 0.3\textdegree
	      respectively and a power-law spectral shape.
\end{enumerate}

The position and sizes of the sources
have been selected so that their contributions overlap. This can be clearly
seen in the significance map shown in the left panel of
Figure~\ref{fig:cube_analysis}. This map was produced with the
\code{ExcessMapEstimator} (see Section~\ref{ssec:gammapy-estimators}) with a
correlation radius of 0.1\textdegree.

We can now fit the same model shapes to the simulated data and retrieve the
best-fit parameters. To check the model agreement, we compute the residual
significance map after removing the contribution from each model. This is done
again via the \code{ExcessMapEstimator}. As can be seen in the middle panel of
Figure~\ref{fig:cube_analysis}, there are no regions above or below 5$\sigma$,
meaning that the models describe the data sufficiently well.

As the example above shows, the 3D analysis allows to characterize the
morphology of the emission and fit it together with the spectral properties of
the source.  Among the advantages that this provides is the ability to
disentangle the contribution from overlapping sources to the same spatial
region. To highlight this, we define a circular RegionGeom of radius
0.5\textdegree~ centered around the position of the point source, which is drawn
in the left panel of Figure~\ref{fig:cube_analysis}. We can now compare the
measured excess counts integrated in that region to the expected relative
contribution from each of the three source models. The result can be seen in the left
panel of Figure~\ref{fig:cube_analysis}.

\begin{figure*}[t]
	\centering
	\includegraphics[width=1.\textwidth]{figures/cube_analysis.pdf}
	\caption{Example 3D analysis for simulated sources using the \cta \irfs. The
		left image shows a significance map where the three simulated sources can be
		seen. The middle figure shows another significance map, but this time after
		subtracting the best-fit model for each of the sources, which are displayed in
		black. The right figure shows the contribution of each source model to the
		circular region of radius 0.5\textdegree drawn in the left image, together with
		the excess counts inside that region. } \label{fig:cube_analysis}
\end{figure*}

Note that all the models fitted also have a spectral component, from which flux
points can be derived in a similar way as described in~\ref{ssec:1d-analysis}.
%\end{figure*}%	\caption{Fermi-LAT TS map in two energy bands} \label{fig:fermi_ts_map}%	\includegraphics[width=1.\textwidth]{figures/fermi_ts_map.pdf}%Ref:~\citep{Stewart2009} \begin{figure*}[t] \centering%\todo{What to do with } Figure~\ref{fig:fermi_ts_map} ?%
