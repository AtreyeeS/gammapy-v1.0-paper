\section{Gammapy project}
\label{sec:gammapy-project}

In this section, we provide an overview of the organization of the Gammapy project. We briefly describe the main roles and responsibilities within the team, as well as the technical infrastructure designed to facilitate the development and maintenance of Gammapy as a high quality software. We use common tools and services for software development of Python open-source projects, code review, testing, package distribution and user support, with a customized solution for a versioned thoroughly tested documentation in the form of user-friendly playable recipes. This section concludes with an outlook on the roadmap for future directions.

\subsection{Organizational structure}
\label{ssec:organizational-structure}

Gammapy is a well-organised international open-source project with a broad developer base and the presence of commitments from strong groups and institutes in the very high-energies astrophysics domain \footnote{\url{https://gammapy.org/team.html}}. The main development roadmaps are discussed and validated by a Coordination Committee, composed of representatives of the main contributing laboratories. This committee is chaired by a Project Manager and his deputy while two Lead Developers manage the development strategy and organise technical activities. Because of this institutionally driven characteristic, the permanent staff and commitment of supporting institutes ensure the continuity of the executive teams. A core team of developers from the contributing laboratories is in charge of the regular development, which benefits from punctual contributions of the community at large.

\subsection{Technical infrastructure}
\label{ssec:technical-infrastructure}

Gammapy follows an open-source and open-contribution development model based on the cloud repository service GitHub. A GitHub organization \textit{gammapy} \footnote{\url{https://github.com/gammapy}} hosts different repositories related with the project. The software codebase may be found in the \textit{gammapy} repository (see Table~\ref{table:codestats:data} and Figure~\ref{fig:codestats:lang} for code lines statistics). We make extensive use of the pull request system to discuss and review code contributions.

\begin{table}
	\import{tables/generated/}{codestats}
	\caption{Coding languages statistics in Gammapy project}
	\label{table:codestats:data}
\end{table}

\begin{figure}[t]
	\centering
	\includegraphics[width=0.5\textwidth]{figures/codestats.pdf}
	\caption{
		Percentage of lines of code in Gammapy project} \label{fig:codestats:lang}
\end{figure}

Several automatized tasks are set as GitHub actions \footnote{\url{https://github.com/features/actions}}, blocking the processes and alerting developers when fails occur. This is the case of the content integration workflow, which monitors the execution of the test coverage suite \footnote{\url{https://pytest.org}} using datasets from the \textit{gammapy-data} repository. Tests scan not only the codebase but also the code snippets present in docstrings of the scripts and in the RST documentation files, as well as in the tutorials provided in the form of Jupyter notebooks. 

Other automatized tasks, executing in the \textit{gammapy-benchmarks} repository, are responsible for numerical validation tests and benchmarks monitoring. Also tasks related with the release process are partially automatized, and every contribution to the codebase repository triggers the documentation building and publishing workflow within the \textit{gammapy-docs} repository (see Sec. ~\ref{ssec:software-distribution} and Sec. ~\ref{ssec:documentation}).

This small ecosystem of interconnected up-to-date repositories, automatized tasks and alerts, is just a part of a bigger set of GitHub repositories, where most of them are related with the project but not necessary for the development of the software (i.e. project webpage, complementary high-energy astrophysics object catalogs, coding sprints and weekly developer calls minutes, contributions to conferences, other digital assets, etc.) Finally, third-party services for code quality metrics are also set and may be found as status shields in the codebase repository.

\subsection{Software distribution}
\label{ssec:software-distribution}

\begin{itemize}
	\item pip, conda, versions
	\item gammapy download
	\item automated release process with Zenodo publishing linking
	\item escape 2020 software platform
	\item ascl
\end{itemize}

\subsection{Documentation}
\label{ssec:documentation}

\begin{itemize}
	\item rst, sphinx, api
	\item versioned notebooks building and testing
	\item versioned binder
\end{itemize}

\subsection{Community and user-support}
\label{ssec:community-and-user-support}

\begin{itemize}
	\item web and on-line documentation 	
	\item slack
	\item github discussions
	\item weekly developers calls
	\item coding sprints 
	\item gammapy recipes repo
	\item mailing lists 
		\begin{verbatim}
		gammapy-coordination-l@in2p3.fr 
		gammapy@googlegroups.com 
		gammapy-cta-l@in2p3.fr
		\end{verbatim}
	\item twitter
\end{itemize}

\subsection{Roadmap}
\label{ssec:roadmap}
\todo{Axel, Regis}

\begin{itemize}
	\item pigs
	\item roadmap
\end{itemize}
