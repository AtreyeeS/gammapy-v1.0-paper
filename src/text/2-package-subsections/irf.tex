\subsection{gammapy.irf}
\label{ssec:gammapy-irf}
\todo{Fabio Pintore}
The {\it gammapy.irf} sub-package contains all classes and functionalities able to handle Instrument Response Functions (\irf) in a wide variety of formats. Usually, \irfs store instrument properties in the form of multi-dimension tables, with quantities expressed in terms of energy (true or reconstructed), off-axis angles, detector coordinates and so on. The main information stored in the common \gammaray \irfs are the effective area (Aeff), energy dispersion (Edisp), point spread function (PSF) and background (BKG). The {\it gammapy.irf} sub-package can open and access specific \irf extensions, interpolate and evaluate the quantities of interest on both energy and spatial axes, convert their format or units in different kinds, plot or write them into output files. In the following, we list the main sub-classes of the sub-package: 

\begin{itemize}  
\item Effective area: \gammapy provides the class {\it EffectiveAreaTable2D} to manage Aeff, which is usually defined in terms of true energy and offset angle. The class functionalities offer the possibility to read from files or to create it from scratch. {\it EffectiveAreaTable2D} can also convert, interpolate, write, and evaluate the effective area for a given energy and offset angles, or even plot the multi-dimensional Aeff table. 

\item Point spread function: \gammapy allows user to treat different kinds of PSFs, in particular, multi-dimensional gaussians ({\it EnergyDependentMultiGaussPSF}), King functions ({\it PSFKing}), or a parametric one. The {\it EnergyDependentMultiGaussPSF} class is able to handle up to three gaussians, defined in terms of amplitudes and sigma given for each true energy and offset angle bin. Similarly, {\it PSFKing takes} into account the gamma and sigma parameters as defined here. The {\it ParametricPSF} allows to create a PSF with a representation different from gaussian(s) or King profile(s). Finally, the user can take advantage from the {\it PSFMap} class, which creates a multi-dimensional map of the PSF in WCS coordinates. At each position, a PSF kernel map ({\it PSFKernel}) provides a PSF as a function of the true energy. The creation of PSF kernel maps, where the PSF is defined for each sky-position, is also given to the user. The latter two can speed up analyses. 

\item Energy dispersion: Edisp, in \iact, is generally given in terms of the so called migration parameter ($\mu$), which is defined as the ratio between the reconstructed energy and the true energy. This ratio should be as close as one and its dispersion can assume the shape of a gaussian (or even more complex distributions). Migration parameter is given at each offset angle and reconstructed energy. The main sub-classes are {\it EnergyDispersion2D}, designed to interpret Edisp, {\it EDispKernelMap}, which builds an Edisp kernel map, i.e. a 4-dimensional WCS map where at each sky-position is associated an Edisp kernel. The latter is a representation of the Edisp as a function of the true energy only thanks to the sub-class {\it EDispKernel}. 

\item Background: the BKG can be represented in \gammapy as either 1) a 2D map ({\it Background2D}) of count rate normalised per steradians and energy at different energies and offset-angles or 2) as rate per steradians and energy, as a function of reconstructed energy and detector coordinates ({\it Background3D}). In the former, the background is expected to follow a radially symmetric shape, while in the latter, it can be more complex.

\end{itemize} 

\begin{figure}
%	\import{code-examples/generated/}{gp_makers}

	\caption{In Fig.~\ref{ig*:minted:irf_examples}, we show some example of Aeff, PSF, Edisp and BKG read and plotted from a typical \irf.}
	\label{ig*:minted:irf_examples}
\end{figure}

