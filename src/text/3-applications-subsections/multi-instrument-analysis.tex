\subsection{Multi instrument analysis}
\label{ssec:multi-instrument-analysis}

\begin{figure*}[t]
	\sidecaption
	\includegraphics[width=0.666\textwidth]{figures/multi_instrument_analysis.pdf}
	\caption{A multi-instrument analysis of the Crab Nebula}
	\label{fig:multi_instrument_analysismulti_instrument_analysis}
\end{figure*}

%% should we mention the joint-crab here or in the intro, where we talk about
%% DL3, multi-instrument analysis and so forth...
%% cite Laura's DL3 HAWC paper with the updated version of the joint-crab spectrum?

In this multi-instrument analysis example we showcase the capabilities of \gammapy
to perform a simultaneous likelihood fit incorporating data from different instruments
and at different levels of reduction.
We estimate the spectrum of the Crab Nebula combining data from \fermi, \magic and \hawc.
Maps of \fermi data are prepared selecting a region of $X^{\circ}$ around the
position of the Crab Nebula applying the same selection criteria of the 3FHL
catalog (7 years of data with energy from $10\,{\rm GeV}$ to $2\,{\rm TeV}$, \citealt{3FHL}).
%% shall we explain exactly which type of Map is created?
The \magic data are two observations of $20\,{\rm min}$ each, chosen from the
dataset used to estimate the performance of the upgraded stereo system \citep{magic_performance}
and already included in \cite{joint_crab}. The observations were taken at small
zenith angles ($<30^{\circ}$) in wobble mode \citep{fomin_1994}, with the source
sitting at an offset of $0.4^{\circ}$ from the FoV center. Their energy range spans
$80\,{\rm GeV} -- 20\,{\rm TeV}$. They are reduced to ON/OFF dataset before being fitted.
\hawc flux points data are estimated in \cite{hawc_crab_2019} with $2.5\,{\rm years}$
of data and span an energy range $300\,{\rm GeV} -- 300\,{\rm TeV}$, and directly read with \gammapy.

\gammapy automatically generates a likelihood including three different types of
terms. Two poissonian likelihoods: one for the \fermi map and one for the ON/OFF
counts, and a $\chi^2$ accounting for the flux points. For \fermi, a
a three-dimensional forward folding of the sky model with the IRF is performed,
in order to compute the predicted counts in each sky-coordinate and energy bin.
For \magic, a one-dimensional forward folding of the spectral model with the IRF
is performed to predict the counts in each estimated energy bin.
A log parabola is fitted to the almost fived decades in energy $10\,{\rm GeV} -- 300\,{\rm TeV}$.
%% is the LP formula specified elsewhere?

The result of the joint fit is displayed in Fig.~\ref{multi_instrument_analysis}.
We remark that the objective of this exercise is illustrative, we display the
flexibility of \gammapy in simultaneously fitting multi-isntrument data even at
different levels of reduction, we do not aim to provide a new measurement of the
Crab Nebula spectrum.

%% eventual part for physical modelling with naima
