\subsection{gammapy.data}
\label{ssec:gammapy-data}
The \code{gammapy.data} sub-package implements the functionality to select,
read and represent DL3 \gammaray data in memory. It provides the main user
interface to access the lowest data level. \gammapy currently only
supports data that is compliant with v0.2 of the \gadf data format.
As both the \gadf data model and \gammapy were initially conceived for
\iact data analysis, DL3 data are typically bundled into individual
observations, which correspond to stable periods of data
aquisitions (typically $20 - 30\,{\rm min}$) for a given instrument.
Each observation is assigned a unique integer ID for reference.

A typical usage example is shown in Figure~\ref{fig*:minted:gp_data}.
First a \code{DataStore} object is created from the path of the data
directory, which contains the DL3 data index file. The \code{DataStore}
object gathers a collection of observations and providing ancillary
files containing information about the telescope observation mode and the
content of the data unit of each file. The \code{DataStore} allows for
selecting a list of observations based on specific filters.

The so-called DL3 files represented by the \code{Observation} class consist
of two types of elements: a list of \gammaray events with relevant physical
quantities for the successive analysis (estimated energy, direction and arrival
times) that is handled by the \code{EventList} class and an instrument
response function (IRF), providing the response of the system, typically
factorised in independent components (see the description in
Sec.~\ref{ssec:gammapy-irf}). The separate handling of event lists and IRFs
additionally allows for data from other \gammaray instruments to be read. For
example, to read \fermi data, the user can read separately their event list
(already compliant with the \gadf specifications) and then find the appropriate
IRF class representing the response functions provided by \fermi, see
Sec.~\ref{ssec:fermi}.
%
\begin{figure}[ht!]
	\import{code-examples/generated/}{gp_data}
	\caption{
        Using \code{gammapy.data} to access DL3 level data with a \code{DataStore} object.
        Individual observations can be accessed by their unique integer observation id number.
        The actual events and instrument response functions can be accessed
        as attributes on the \code{Observation} object, such as \code{.events}
        or \code{.aeff} for the effective area information.
    }
	\label{fig*:minted:gp_data}
\end{figure}
%
