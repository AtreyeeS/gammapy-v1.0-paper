\subsection{gammapy.makers}
\label{ssec:gammapy-makers}
%
\begin{figure}
	\import{code-examples/generated/}{gp_makers}

	\caption{
        Using \code{gammapy.makers} to reduce DL3 level data into a
		\code{MapDataset}. All \code{Maker} classes take
        the configuration on initialisation of the class.
    }
	\label{ig*:minted:gp_makers}
\end{figure}
%
The data reduction step includes all tasks required to process and prepare
\gammaray data from the DL3 level to modeling and fitting. The \code{gammapy.makers} sub-package
contains the various classes and functions required to do so. First, events are
binned and IRFs are interpolated and projected onto the chosen analysis
geometry. This produces counts, exposure, background, psf and energy dispersion
maps. The \code{MapDatasetMaker} and \code{SpectrumDatasetMaker} are
responsible for this task, see Fig~\ref{ig*:minted:gp_makers}.

Because the background models suffer from strong uncertainties it is required
to correct them from the data themselves. Several techniques are commonly used
in gamma-ray astronomy such as field-of-view background normalization or
background measurement in reflected regions regions, see~\cite{Berge07}.
Specific \code{Makers} such as the \code{FoVBackgroundMaker} or the
\code{ReflectedRegionsBackgroundMaker} are in charge of this step.

Finally, to limit other sources of systematic uncertainties, a data validity
domain is determined by the \code{SafeMaskMaker}. It can be used to limit the
extent of the field of view used or to limit the energy range to e.g., a domain
where the energy reconstruction bias is below a given value.
