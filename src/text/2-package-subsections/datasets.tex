\subsection{gammapy.datasets}
\label{ssec:gammapy-datasets}
\todo{Atreyee Sinha}
DL4 level data

\begin{figure}

	\import{code-examples/generated/}{gp_datasets}
	\caption{Using gammapy.data to access DL3 level data with a DataStore}
	\label{fig*:minted:gp_datasets}
\end{figure}

Reduced data, IRFs, and models are bundled together to create a Data Level 4 (DL4). This is provide in the Dataset class which computes a likelihood to interface with the Fit class to perform a fitting. Corresponding to the different analysis types, five types of datasets are supported in gammapy (\ref{table:datasets_types}). 3D data cubes (with two spatial and one energy) are represented by MapDatasets, whereas SpectrumDatasets contain binned energy information for a given on region. 2D images are a special case of MapDatasets, with only one bin in energy. The FluxPointsDataset is used to fit pre-computed flux points when no convolution with IRFs are needed.

\begin{table*}[!ht]
	\centering
	\begin{tabular}{|l|l|l|l|l|l|l|}
		\hline
		Dataset Type         & Data Type & Reduced IRFs                      & Geometry              & Additional Quantities                    & Fit Statistic & Purpose                                                                         \\ \hline
		MapDataset           & counts    & background, psf, edisp, exposure, & WcsGeom or RegionGeom & --                                       & cash          & A full 3D analysis with a field of view background                              \\ \hline
		MapDatasetOnOff      & counts    & psf, edisp, exposure              & WcsGeom               & acceptance, acceptance\_off, counts\_off & wstat         & Sky Images with a Ring Background / Background estimation from off observations \\ \hline
		SpectrumDataset      & counts    & background, edisp, exposure       & RegionGeom            & --                                       & cash          & 1D spectral extraction with the FoV background                                  \\ \hline
		SpectrumDatasetOnOff & counts    & edisp, exposure                   & RegionGeom            & acceptance, acceptance\_off, counts\_off & wstat         & 1D spectral extractions using Reflected regions                                 \\ \hline
		FluxPointsDataset    & flux      & None                              & None                  & --                                       & chi2          & Fitting of precomputed flux points                                              \\ \hline
	\end{tabular}
	\caption{Some caption.}
	\label{table:datasets_types}
\end{table*}

Multiple datasets, or the same or different type, can be contained in a Datasets container. The Datasets class adds the likelihood from its constituent members to compute the full likelihood for a joint fitting.

The total number of predicted counts for a dataset are computed as the sum of the predicted counts from all source components, plus the expected counts from the residual hadronic background. In case of normal datasets, the predicted counts from the hadronic background are computed directly from the model in reconstructed energy and spatial coordinates, whereas for OnOff datasets, the background is estimated from real off counts. The predicted counts from a source are obtained by forward folding with the IRFs, assuming that the IRFs are independent of one another.
