\subsection{1D Analysis}
\label{ssec:1d-analysis}
One of the most common analysis cases in gamma-ray astronomy is measuring the
spectrum of a source in a given region defined on the sky, in conventional
astronomy also called aperture photometry. The spectrum is typically measured
in two steps: first a parametric spectral model is fitted to the data and
secondly flux points are computed in a pre-defined set of energy bins. The
result of such an analysis performed on three simulated CTA observations is
shown in Fig.~\ref{fig:cta_galactic_center}. In this case the spectrum was
measured in a circular aperture centered on the Galactic Center, in
\gammaray~astronomy often called "on region". For such analysis the users first
chooses a region of interest and energy binning, both defined by a
\code{RegionGeom}. In a second step the events and instrument response are binned
into maps of this geometry, by the \code{SpectrumDatasetMaker}. All the data and
reduced instrument response are bundled into a \code{SpectrumDataset}. To estimate
the expected background in the on region a "reflected regions" background
method was used~\cite{Berge07}, represented in \gammapy by the
\code{ReflectedRegionsBackgroundMaker} class. The resulting reflected regions are
illustrated for all three observations on top of the map of counts. After
reduction of the data it was modelled using a forward-folding method and a
power-law spectral shape, using the \code{PowerLawSpectralModel} and \code{Fit} class.
Based on this best fit model the final flux points and corresponding
log-likelihood profiles are computed using the \code{FluxPointsEstimator}.

\begin{figure*}[t]
	\centering
	\includegraphics[width=1.\textwidth]{figures/cta_galactic_center.pdf}
	\caption{
		Example spectral analysis of the Galactic Center for three simulated CTA
		observations. The left image shows the maps of counts with the measurement
		region and background regions overlaid in different colors. The right image
		shows the resulting spectral points and their corresponding log-likelihood
		profiles.} \label{fig:cta_galactic_center} \end{figure*}

