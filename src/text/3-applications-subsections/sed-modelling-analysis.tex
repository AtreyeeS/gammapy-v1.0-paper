\subsection{Broadband SED Modeling}
\label{ssec:broadband-sed-modeling}
By combining \gammapy with astrophysical modelling codes users can also fit
astrophysical spectral models to \gammaray data. In \gammaray
astronomy what typically observes two radiation production
mechanisms., the so called hadronic and leptonic scenarios.
There are mutiple Python packages that are able to model
the \gammaray emission, given a physical scenario. Among those
packages are  \cite{agnpy}, \cite{naima}, \cite{jetset} or \cite{gamera}.
Tyically those emission models predict broadband emission from
radio waves up to the very high energy \gammaray range.
By relying on the multiple dataset types in \gammapy those
data can be combined to constrain such a broadband emission model.
\gammapy provides a built-in \code{NaimaSpectralModel} which allows
users to wrap a given astrophysical emission model from the
Naima package and fit it directly to \gammaray data.
