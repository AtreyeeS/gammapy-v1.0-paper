\section{Gammapy project}
\label{sec:gammapy-project}

In this section, we provide an overview of the organization of the Gammapy project. We briefly describe the main roles and responsibilities, as well as the technical infrastructure designed to facilitate the development and maintenance of Gammapy as a high quality open source software. We also describe how we tackle with the software distribution, our solution for a versioned thoroughly tested documentation in the form of user-friendly playable recipes, and the tools we use to foster the Gammapy community and to provide user support. This section concludes with an outlook on the roadmap for future directions.

\subsection{Organizational structure}
\label{ssec:organizational-structure}

Gammapy is a well-organised international open-source project with a broad developer base and the presence of commitments from strong groups and institutes in the very high-energies astrophysics domain \footnote{\url{https://gammapy.org/team.html}}. The main development roadmaps are discussed and validated by a Coordination Committee, composed of representatives of the main contributing laboratories. This committee is chaired by a Project Manager and his deputy while two Lead Developers manage the development strategy and organise technical activities. Because of this institutionally driven characteristic, the permanent staff and commitment of supporting institutes ensure the continuity of the executive teams. A core team of developers from the contributing laboratories is in charge of the regular development, which benefits from punctual contributions of the community at large.

\subsection{Technical infrastructure}
\label{ssec:technical-infrastructure}

\begin{itemize}
	\item GitHub repositories 
	\begin{itemize}
		\item gammapy (describe gammapy codebase)
		\item docs
		\item web
		\item data
		\item benchmarks
		\item recipes
		\item extra
		\item meetings
		\item conferences contributions
		\item catalogs? (i.e. gamma-cat)
		\item paper
	\end{itemize}
	\item github actions and automated tasks 
	\begin{itemize}
		\item ci
		\item docs
		\item benchmarks-slack
		\item codemeta.json and zenodo
		\item release-web
	\end{itemize}	
	\item pytest and code coverage in ci
	\item tests on docstrings and on code in rst files
	\item code quality services (codecov, lgtm, codacy)
	\item validation and benchmarks -validation as online appendix...?
\end{itemize}

\begin{table}
	\import{tables/generated/}{codestats}
	\caption{Coding languages statistics in Gammapy project}
	\label{table:codestats:data}
\end{table}

\begin{figure}[t]
	\centering
	\includegraphics[width=0.5\textwidth]{figures/codestats.pdf}
	\caption{
		Percentage of lines of code in Gammapy project} \label{fig:codestats:lang}
\end{figure}

\subsection{Software distribution}
\label{ssec:software-distribution}

\begin{itemize}
	\item pip, conda, versions
	\item gammapy download
	\item automated release process with Zenodo publishing linking
	\item escape 2020 software platform
	\item ascl
\end{itemize}

\subsection{Documentation}
\label{ssec:documentation}

\begin{itemize}
	\item rst, sphinx, api
	\item versioned notebooks building and testing
	\item versioned binder
\end{itemize}

Figure: Screenshot of Jupyter notebook or docs with notebook, could show the interactive maps view
\begin{verbatim}
m = Map.read(“diffuse.fits”)
m.plot_interactive()
\end{verbatim}


\subsection{Community and user-support}
\label{ssec:community-and-user-support}

\begin{itemize}
	\item weekly developers calls
	\item coding sprints 
	\item slack
	\item github discussions
	\item gammapy recipes repo
	\item mailing lists 
		\begin{verbatim}
		gammapy-coordination-l@in2p3.fr 
		gammapy@googlegroups.com 
		gammapy-cta-l@in2p3.fr
		\end{verbatim}
	\item twitter
\end{itemize}

\subsection{Roadmap}
\label{ssec:roadmap}
\todo{Axel, Regis}

\begin{itemize}
	\item pigs
	\item roadmap
\end{itemize}
