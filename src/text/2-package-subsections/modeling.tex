\subsection{gammapy.modeling}
\label{ssec:gammapy-modeling}
\todo{Quentin Remy}

gammapy.modeling contains all the functionality related to modeling and fitting data. This includes spectral, spatial and temporal model classes, as well as the fit and parameter API.

\subsection{Models}
\label{ssec:models}

The models are grouped into the following categories:
\begin{itemize}
\item SpectralModel: models to describe spectral shapes of sources
\item SpatialModel: models to describe spatial shapes (morphologies) of sources
\item TemporalModel: models to describe temporal flux evolution of sources, such as light and phase curves
\end{itemize}
The models follow a naming scheme which contains the category as a suffix to the class name.

The  Spectral Models include a special class of Normed models, which have a dimension-less normalisation. These spectral models feature a norm parameter instead of amplitude and are named using the NormSpectralModel suffix. They must be used along with another spectral model, as a multiplicative correction factor according to their spectral shape. They can be typically used for adjusting template based models, or adding a EBL correction to some analytic model. The analytic Spatial models are all normalized such as they integrate to unity over the sky but the template Spatial models may not, so in that special case they have to be combined with a NormSpectralModel.

The SkyModel is a factorised model that combine the spectral, spatial and temporal model components (by default the spatial and temporal components are optional).
SkyModel objects represents additive emission components, usually sources or diffuse emission, although a single source can also be modeled by multiple components.
To handle list of multiple SkyModel components, Gammapy has a Models class.

The model gallery provides a visual overview of the available models in Gammapy.
Most of the analytic models  commonly used in gamma-ray astronomy are built-in. We also offer a wrapper to radiative models implemented in the Naima package \cite{naima}. The modeling framework can be easily extended with user-defined models. For example agnpy models that describe leptonic radiative processes in jetted Active Galactic Nuclei (AGN) can wrapped into gammapy \citep[see section3.5 of ][]{2021arXiv211214573N} .

\begin{figure}
	\import{code-examples/generated/}{gp_models}
	\caption{Using gammapy.modeling.models}
	\label{fig*:minted:gp_models}
\end{figure}


\subsection{Fit}
\label{ssec:fit}

The Fit class provides methods to fit, i.e. optimise parameters and estimate parameter errors and correlations. It interfaces with a Datasets object, which in turn is connected to a Models object containing the model parameters in its Parameters object.  Models can be unique for a given dataset, or contribute to multiple datasets and thus provide links, allowing e.g. to do a joint fit to multiple IACT datasets, or to a joint IACT and \textit{Fermi}-LAT dataset. Many examples are given in the tutorials.

The Fit class provides a uniform interface to multiple fitting backends: “minuit” \citep{iminuit}, “scipy”, \citep{2020SciPy-NMeth}, and “sherpa” \citep{sherpa-2005,sherpa-2011}.
Note that, for now, covariance matrix and errors are computed only for the fitting with MINUIT. However depending on the problem other optimizers can better perform, so sometimes it can be useful to run a pre-fit with alternative optimization methods.
In future we plan to extend the supported Fit backend, including for example MCMC solutions.  \footnote{a prototype is available in gammapy-recipes, \url(https://gammapy.github.io/gammapy-recipes/_build/html/notebooks/mcmc-sampling-emcee/mcmc_sampling.html)}
