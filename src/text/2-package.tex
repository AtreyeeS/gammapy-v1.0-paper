\section{Gamma-ray Data Analysis}
\label{sec:gammaray-data-analysis}
%
\begin{figure*}[ht!]
	\centering
	\includegraphics[width=1.\textwidth]{figures/irfs.pdf}
	\caption{
		Example instrument response functions of some experiments and
        observatories for which \gammapy can analyse data. The \cta IRFs
        are from the prod5 production. The \hess IRFs are from the DL3 DR1,
        using observation ID 033787. The point spread function shows the 65\&
        containment radius of the PSF. The \fermi IRFs are from \textit{pass8}.
        \todo{Add more instruments? HAWC? MAGIC?}
    }
	\label{fig:irfs}
\end{figure*}
%
% This might not be the bast place compared to introduction
The data reduction process in \gammaray astronomy is usually split into two parts.
The first one deals with the data processing from camera measurement, calibration, event
reconstruction and selection to yield a list of reconstructed \gammaray event candidates.
This sequence, sometimes referred to as low-level analysis, is usually very specific to
a given observation technique and even to a given instrument.

The other sequence, referred to as high-level analysis, deals with the extraction of physical
quantities related to \gammaray sources and the production of high-level products such as spectra,
lightcurves and catalogs. The methods and tools applied here are more generic and are broadly
shared across the field. They also frequently imply joint analysis of multi-instrument data.
To extract physically relevant information, the measured data are usually compared to a
model of the expected \gammaray emitters in the instrument field-of-view using statistical
techniques such as maximum likelihood.

We can write the expected number of detected events at measured position $p$ and energy $E$:
\begin{align}
   N(p, E) {\rm d}p {\rm d}E = &t_{\rm obs} \int_{E_{\rm true}} \int_{p_{\rm true}}  R(p, E|p_{\rm true}, E_{\rm true})\\
   &\times \Phi(p_{\rm true}, E_{\rm true}) {\rm d}E_{\rm true} {\rm d}p_{\rm true}
\end{align}
where
\begin{itemize}
\item $R(p, E| p_{\rm true}, E_{\rm true})$ is the instrument response
\item $\Phi(p_{\rm true}, E_{\rm true})$ is the sky flux model
\item $t_{\rm obs}$ is the observation time
\end{itemize}

A common assumption is that the instrument response can be simplified as the product
of three independent functions:

\begin{align}
   R(p, E|p_{\rm true}, E_{\rm true}) = &A_{\rm eff}(p_{\rm true}, E_{\rm true}) \times\\
    &PSF(p|p_{\rm true}, E_{\rm true}) \times\\
    &E_{\rm disp}(E|p_{\rm true}, E_{\rm true})
\end{align}
where:
\begin{itemize}
\item $A_{\rm eff}(p_{\rm true}, E_{\rm true})$ is the effective collection area of the detector. It is the product
  of the detector collection area times its detection efficiency at true energy $E_{\rm true}$ and position $p_{\rm true}$.
\item $PSF(p|p_{\rm true}, E_{\rm true})$ is the point spread function. It gives the probability of
  measuring a direction $p$ when the true direction is $p_{\rm true}$ and the true energy is $E_{\rm true}$.
  \gammaray instruments consider the probability density of the angular separation between true and reconstructed directions
  $\delta p = p_{\rm true} - p$, i.e. $PSF(\delta p|p_{\rm true}, E_{\rm true})$.
\item $E_{\rm disp}(E|p_{\rm true}, E_{\rm true})$ is the energy dispersion. It gives the probability to
  reconstruct the photon at energy $E$ when the true energy is $E_{\rm true}$ and the true position :$p_{\rm true}$.
  \gammaray instruments consider the probability density of the migration $\mu=\frac{E}{E_{\rm true}}$,
  i.e. $E_{\rm disp}(\mu|p_{\rm true}, E_{\rm true})$.
\end{itemize}

\gammaray data at the Data Level 3 therefore consists in lists of gamma-like events and their
corresponding instrument response functions (IRFs). The latter include the aforementioned
effective area, point spread function (PSF), energy dispersion and residual hadronic background.
The handling of DL3 data is performed by classes and methods
in the gammapy.data  (see \ref{ssec:gammapy-data}) and the gammapy.irf
(see \ref{ssec:gammapy-irf}) subpackages.

The first step in the analysis is the selection and extraction of observations
based of their meta data including information such as pointing direction, observation
time and observation conditions.

The next step of the analysis is the data reduction where all observation events and instrument
responses are projected onto a user-defined geometry.A typical geometry consists in a spectral representation with a measured
energy axis, and in a spatial representation, either a coordinates system  with a projection
(for 3-dimensional or cube analysis) or a region on the sky (for regular spectral analysis).
The gammapy.maps subpackage provides general multidimensional geometry objects
(Geom) and the associated data structures (Maps), see \ref{ssec:gammapy-maps}.

All observation events and instrument responses are projected onto the
user defined geometry. Because residual hadronic background models can be subject
to significant uncertainties, background correction must be applied,
such as the ring or the field-of-view background techniques or
background measurements must be performed within, e.g. reflected regions~\citep{Berge07}.
Parts of the data with high associated IRF systematics must also be excluded by defining
a "safe" data range. These data reduction steps are performed by classes and functions
implemented in the gammapy.makers subpackage (see \ref{ssec:gammapy-makers}).

The counts data and the reduced IRFs in the form of maps are bundled into dataset objects
that represent the data level 4 (DL4). They can be written to
disk, in a format specific to \gammapy to allow users to read them back at any time later
for modeling and fitting.

This latter step datasets classes bundle reduced data in form of maps, reduced IRFs, models and
fit statistics. Different sub-classes support different analysis methods
and fit statistics (e.g. Poisson statistics with known background or
with OFF background measurements). The datasets are used to perform joint-likelihood
fitting allowing to combine different measurements, e.g. from different observations
but also from different instruments or event classes. They can also be used for binned
simulation as well as event sampling to simulate DL3 events data.

The next step is then typically to model and fit the datasets, either
individually, or in a joint likelihood analysis. For this purpose \gammapy
provides a uniform interface to multiple fitting backends. It also provides
a variety of :ref:`built in models <model-gallery>`. This includes spectral,
spatial and temporal model classes to describe the \gammaray emission in the sky.
Where spectral models can be simple analytical models or more complex ones from radiation
mechanisms of accelerated particle populations (e.g. inverse Compton or $\pi^{o}$ decay).
Independently or subsequently to the global modelling, the data can be
re-grouped to compute flux points, light curves and flux as well as significance
maps in energy bands.

\section{\gammapy Package}
\label{sec:gammapy-package}
\subsection{Overview}
\label{ssec:overview}
%
\begin{figure*}[ht!]
	\centering
	\includegraphics[width=1.\textwidth]{figures/data_flow.pdf}
	\caption{
		\gammapy sub-package structure and data analysis workflow. The top row
        defines the groups for the different data levels and reduction steps
        from raw gamma-like events on the left, to high level science products
        on the right. The direction of the data flow is illustrated with the
        grey arrows. The gray folder icons represent the different sub-packages
        in \gammapy and their names. Below each icon there is a list of the most
        important objects defined in the sub-package.
    }
	\label{fig:data_flow}
\end{figure*}
%
The \gammapy package is structured into multiple sub-packages. The definition
of the content of the different sub-packages follows mostly the stages in the
data reduction workflow described in the previous section. Sub-packages
either contain data structures representing data at different data reduction
levels or contain algorithms to transition between the different data reduction
levels.

Figure~\ref{fig:data_flow} shows an overview of the different sub-packages and
their relation to each other. The \code{gammapy.data} and \code{gammapy.irf}
sub-packages define data objects to represent DL3 data, such as
event lists and instrument response functions as well as functionality
to the DL3 data from disk into memory. The \code{gammapy.makers} sub-package
contains the functionality to reduce the DL3 data to binned maps.
Binned maps and datasets, which represent a collection of binned
maps are defined in the \code{gammapy.maps} and \code{gammapy.datasets}
sub-packages respectively. Parametric models, which are defined in
\code{gammapy.modeling}, are used to jointly model a combination
of datasets for example to create source catalogs. Estimator classes,
which are contained in \code{gammapy.estinators} are used to
compute higher level science products such as flux and signficance maps,
light curves or flux points. Finally there is a \code{gammapy.analysis}
sub-package which provides a high level interface for executing analyses
defined from configuration files. In the following sections we will
introduce all sub-packages and their functionality in more detail.


\subsection{gammapy.data}
\label{ssec:gammapy-data}
The \code{gammapy.data} sub-package implements the functionality to select,
read and represent DL3 \gammaray data in memory. It provides the main user
interface to access the lowest data level. \gammapy currently only
supports data that is compliant with v0.2 of the \gadf data format.
As both the \gadf data model and \gammapy were initially conceived for
\iact data analysis, DL3 data are typically bundled into individual
observations, which correspond to stable periods of data
aquisitions (typically $20 - 30\,{\rm min}$) for a given instrument.
Each observation is assigned a unique integer ID for reference.

A typical usage example is shown in Figure~\ref{fig*:minted:gp_data}.
First a \code{DataStore} object is created from the path of the data
directory, which contains the DL3 data index file. The \code{DataStore}
object gathers a collection of observations and providing ancillary
files containing information about the telescope observation mode and the
content of the data unit of each file. The \code{DataStore} allows for
selecting a list of observations based on specific filters.

The so-called DL3 files represented by the \code{Observation} class consist
of two types of elements: a list of \gammaray events with relevant physical
quantities for the successive analysis (estimated energy, direction and arrival
times) that is handled by the \code{EventList} class and an instrument
response function (IRF), providing the response of the system, typically
factorised in independent components (see the description in
Sec.~\ref{ssec:gammapy-irf}). The separate handling of event lists and IRFs
additionally allows for data from other \gammaray instruments to be read. For
example, to read \fermi data, the user can read separately their event list
(already compliant with the \gadf specifications) and then find the appropriate
IRF class representing the response functions provided by \fermi, see
Sec.~\ref{ssec:fermi}.
%
\begin{figure}[ht!]
	\import{code-examples/generated/}{gp_data}
	\caption{
        Using \code{gammapy.data} to access DL3 level data with a \code{DataStore} object.
        Individual observations can be accessed by their unique integer observation id number.
        The actual events and instrument response functions can be accessed
        as attributes on the \code{Observation} object, such as \code{.events}
        or \code{.aeff} for the effective area information.
    }
	\label{fig*:minted:gp_data}
\end{figure}
%

\subsection{gammapy.irf}
\label{ssec:gammapy-irf}
\todo{Fabio Pintore}
The {\it gammapy.irf} sub-package contains all classes and functionalities able to handle Instrument Response Functions (\irf) in a wide variety of formats. Usually, \irfs store instrument properties in the form of multi-dimension tables, with quantities expressed in terms of energy (true or reconstructed), off-axis angles, detector coordinates and so on. The main information stored in the common \gammaray \irfs are the effective area (Aeff), energy dispersion (Edisp), point spread function (PSF) and background (BKG). The {\it gammapy.irf} sub-package can open and access specific \irf extensions, interpolate and evaluate the quantities of interest on both energy and spatial axes, convert their format or units in different kinds, plot or write them into output files. In the following, we list the main sub-classes of the sub-package: 

\begin{itemize}  
\item Effective area: \gammapy provides the class {\it EffectiveAreaTable2D} to manage Aeff, which is usually defined in terms of true energy and offset angle. The class functionalities offer the possibility to read from files or to create it from scratch. {\it EffectiveAreaTable2D} can also convert, interpolate, write, and evaluate the effective area for a given energy and offset angles, or even plot the multi-dimensional Aeff table. 

\item Point spread function: \gammapy allows user to treat different kinds of PSFs, in particular, multi-dimensional gaussians ({\it EnergyDependentMultiGaussPSF}), King functions ({\it PSFKing}), or a parametric one. The {\it EnergyDependentMultiGaussPSF} class is able to handle up to three gaussians, defined in terms of amplitudes and sigma given for each true energy and offset angle bin. Similarly, {\it PSFKing takes} into account the gamma and sigma parameters as defined here. The {\it ParametricPSF} allows to create a PSF with a representation different from gaussian(s) or King profile(s). Finally, the user can take advantage from the {\it PSFMap} class, which creates a multi-dimensional map of the PSF in WCS coordinates. At each position, a PSF kernel map ({\it PSFKernel}) provides a PSF as a function of the true energy. The creation of PSF kernel maps, where the PSF is defined for each sky-position, is also given to the user. The latter two can speed up analyses. 

\item Energy dispersion: Edisp, in \iact, is generally given in terms of the so called migration parameter ($\mu$), which is defined as the ratio between the reconstructed energy and the true energy. This ratio should be as close as one and its dispersion can assume the shape of a gaussian (or even more complex distributions). Migration parameter is given at each offset angle and reconstructed energy. The main sub-classes are {\it EnergyDispersion2D}, designed to interpret Edisp, {\it EDispKernelMap}, which builds an Edisp kernel map, i.e. a 4-dimensional WCS map where at each sky-position is associated an Edisp kernel. The latter is a representation of the Edisp as a function of the true energy only thanks to the sub-class {\it EDispKernel}. 

\item Background: the BKG can be represented in \gammapy as either 1) a 2D map ({\it Background2D}) of count rate normalised per steradians and energy at different energies and offset-angles or 2) as rate per steradians and energy, as a function of reconstructed energy and detector coordinates ({\it Background3D}). In the former, the background is expected to follow a radially symmetric shape, while in the latter, it can be more complex.

\end{itemize} 

\begin{figure}
%	\import{code-examples/generated/}{gp_makers}

	\caption{In Fig.~\ref{ig*:minted:irf_examples}, we show some example of Aeff, PSF, Edisp and BKG read and plotted from a typical \irf.}
	\label{ig*:minted:irf_examples}
\end{figure}


\subsection{gammapy.maps}
\label{ssec:gammapy-maps}
\todo{Laura Olivera-Nieto}
The \verb|gammapy.maps| sub-package provides classes for representing data structures associated with a set of coordinates or a region on a sphere. In addition it allows to handle an arbitrary number of non-spatial data dimensions, such as time or energy. It is organized around three types of structures: geometries, sky-maps and map axes, which inherit from the base classes Geom, Map and MapAxis respectively. 

The geometry defines the pixelization scheme and map boundaries. It also provides methods to transform between sky and pixel coordinates. Maps consist of a geometry instance together with a data array containing the corresponding map values. Map axes contain a sequence of ordered values which define bins on a given dimension, spatial or not. Map axes can have physical units attached to them, as well as non-linear step sizes.

The sub-package provides a uniform API for the FITS World Coordinate System (WCS)~\citep{Calabretta2002}, the HEALPix pixelization~\citep{Gorski2005} and region-based data structures:

% itemize because it helps me write, could also just be paragraphs...
\begin{itemize}
	\item \textbf{WCS: } The FITS WCS pixelization supports a different number of projections to represent celestial spherical coordinates in a regular rectangular grid.
	\item \textbf{HEALPix: } This pixelization provides a subdivision of a sphere in which each pixel covers the same surface area as every other pixel. As a consequence, however, pixel shapes are no longer rectangular, or regular. Gammapy provides limited support to HEALPix-based maps, relying in some cases on projections to a WCS grid.
	\item \textbf{Region geometries: } In this case, instead of a fine spatial grid on a given sky region, the spatial dimension is reduced to a single bin with an arbitrary shape, describing a region in the sky with that same shape. Typically they are is used together with a non-spatial dimension, for example an energy axis to represent how a quantity varies with energy on a given region of the sky.
\end{itemize}






\begin{figure}
	\import{code-examples/generated/}{gp_maps}
	\caption{Using gammapy.data to access DL3 level data with a DataStore}
	\label{codeexample:data}
\end{figure}

\subsection{gammapy.datasets}
\label{ssec:gammapy-datasets}
\todo{Atreyee Sinha}

The end product of the data reduction process described in \ref{ssec:gammapy-makers} are a set of binned counts, background and IRF maps, at the DL4 level. The gammapy.datasets subpackage contains classes to bundle together binned data along with associated models and the likelihood, which provides an interface to the Fit class \ref{ssec:gammapy-modeling} for modeling and fitting purposes. Figure \ref{fig*:minted:gp_datasets} Depending upon the type of analysis (3D or only spectral), and the associated statistics (Cash, wstat or chi2), different types of Datasets are supported. The predicted counts are computed by convolution of the models with the associate IRFs. Fitting of precomputed flux points is enabled through FluxPointsDatasets. Multiple datasets of same or different types can be  bundled together in Datasets, where the likelihood from each constituent member is added, thus facilitating joint fitting across different observations, and even different instruments across different wavelengths. Datasets also provide functionalities for manipulating reduced data, eg: stacking, sub-grouping, plotting, etc. Users can also create their customized datasets for implementing modified likelihood methods. 

\begin{figure}
	\import{code-examples/generated/}{gp_datasets}
	\caption{Using gammapy.datasets for modelling and fitting}
	\label{fig*:minted:gp_datasets}
\end{figure}


\subsection{gammapy.makers}
\label{ssec:gammapy-makers}
\todo{Regis Terrier}

The data reduction contains all tasks required to process and prepare data at
the DL3 level for modeling and fitting. The \texttt{gammapy.makers} sub-package
contains the various classes and functions required to do so. First, events are
binned and IRFs are interpolated and projected onto the chosen analysis
geometry. This produces counts, exposure, background, psf and energy dispersion
maps. The \texttt{MapDatasetMaker} and \texttt{SpectrumDatasetMaker} are
responsible for this task, see Fig~\ref{ig*:minted:gp_makers}.

Because the background models suffer from strong uncertainties it is required
to correct them from the data themselves. Several techniques are commonly used
in gamma-ray astronomy such as field-of-view background normalization or
background measurement in reflected regions regions, see~\cite{Berge07}.
Specific \texttt{Makers} such as the \texttt{FoVBackgroundMaker} or the
\texttt{ReflectedRegionsBackgroundMaker} are in charge of this step.

Finally, to limit other sources of systematic uncertainties, a data validity
domain is determined by the \texttt{SafeMaskMaker}. It can be used to limit the
extent of the field of view used or to limit the energy range to e.g., a domain
where the energy reconstruction bias is below a given value.

\begin{figure}
	\import{code-examples/generated/}{gp_makers}

	\caption{Using gammapy.makers to reduce DL3 level data into a
		\texttt{Dataset}.}
	\label{ig*:minted:gp_makers}
\end{figure}

\subsection{gammapy.modeling}
\label{ssec:gammapy-modeling}
\todo{Quentin Remy}

gammapy.modeling contains all the functionality related to modeling and fitting
data. This includes spectral, spatial and temporal model classes, as well as
the fit and parameter API.

\subsection{Models}
\label{ssec:models}

The models are grouped into the following categories:

\begin{itemize}
	\item SpectralModel: models to describe spectral shapes of sources
	\item SpatialModel: models to describe spatial shapes (morphologies) of sources
	\item TemporalModel: models to describe temporal flux evolution of sources, such as
	      light and phase curves

\end{itemize}

The models follow a naming scheme which contains the category as a suffix to
the class name.

The  Spectral Models include a special class of Normed models, which have a
dimension-less normalisation. These spectral models feature a norm parameter
instead of amplitude and are named using the NormSpectralModel suffix. They
must be used along with another spectral model, as a multiplicative correction
factor according to their spectral shape. They can be typically used for
adjusting template based models, or adding a EBL correction to some analytic
model. The analytic Spatial models are all normalized such as they integrate to
unity over the sky but the template Spatial models may not, so in that special
case they have to be combined with a NormSpectralModel.

The SkyModel is a factorised model that combine the spectral, spatial and
temporal model components (by default the spatial and temporal components are
optional). SkyModel objects represents additive emission components, usually
sources or diffuse emission, although a single source can also be modeled by
multiple components. To handle list of multiple SkyModel components, Gammapy
has a Models class.

The model gallery provides a visual overview of the available models in
Gammapy. Most of the analytic models  commonly used in gamma-ray astronomy are
built-in. We also offer a wrapper to radiative models implemented in the Naima
package~\cite{naima}. The modeling framework can be easily extended with
user-defined models. For example agnpy models that describe leptonic radiative
processes in jetted Active Galactic Nuclei (AGN) can wrapped into
gammapy~\citep[see section3.5 of ][]{2021arXiv211214573N} .

\begin{figure}
	\import{code-examples/generated/}{gp_models}
	\caption{Using gammapy.modeling.models}
	\label{fig*:minted:gp_models}
\end{figure}

\subsection{Fit}
\label{ssec:fit}

The Fit class provides methods to fit, i.e. optimise parameters and estimate
parameter errors and correlations. It interfaces with a Datasets object, which
in turn is connected to a Models object containing the model parameters in its
Parameters object.  Models can be unique for a given dataset, or contribute to
multiple datasets and thus provide links, allowing e.g. to do a joint fit to
multiple IACT datasets, or to a joint IACT and \textit{Fermi}-LAT dataset. Many
examples are given in the tutorials.

The Fit class provides a uniform interface to multiple fitting backends:
“minuit”~\citep{iminuit}, “scipy”,~\citep{2020SciPy-NMeth}, and
“sherpa”~\citep{sherpa-2005,sherpa-2011}. Note that, for now, covariance matrix
and errors are computed only for the fitting with MINUIT. However depending on
the problem other optimizers can better perform, so sometimes it can be useful
to run a pre-fit with alternative optimization methods. In future we plan to
extend the supported Fit backend, including for example MCMC solutions.
\footnote{a prototype is available in gammapy-recipes,
	\url{https://gammapy.github.io/gammapy-recipes/_build/html/notebooks/mcmc-sampling-emcee/mcmc_sampling.html}
}

\subsection{gammapy.estimators}
\label{ssec:gammapy-estimators}
\todo{Axel Donath}
By fitting parametric models to the data, the total integrated
flux and overall temporal, spectral and morphological shape of the
\gammaray emission can be constrained. In many cases it useful
to make a more detailed follow-up analysis by measuring the
flux in smaller spectral, temporal or spatial bins. This
possibly reveals more detailed emission features, which
are relevant for studying correlation with counterpart emissions.

The \code{gammapy.estimators} sub-module features methods to compute flux
points, light curves, flux maps, flux profiles from data.
The basic method for all these measurements is equivalent.
The initial fine bins of \code{MapDataset} are grouped into
larger bins. The norm of the best fit "reference" spectral
model is fitted ("scaled") in the  restricted data range only.

The base class of all algorithms is the \code{Estimator}  class.

Signifiance is estimated based on hypothesis
testing against the background only model. This allows
to assign a significance to the given flux bin.
In addition to the best fit flux norm all estimators compute
quantities corresponding to this flux. This includes
the predicted number of total, signal and background
counts per flux bin. The total fit statistics
of the best fit model, the fit statistics of the
null hypothesis and the difference between both,
the so-called TS value.
From the TS value the significance of the
significance of the measured signal and associated flux
can be derived.

Optionally it can also compute more advanced quantities
such as assymmetric flux errors, flux upper limits
and one dimensional profiles of the fit statistic,
which show how the likelihood functions varies with
the flux norm parameter around the fit minimum.
This information is useful for inspecting the quality
of the fit, where assumptoticcally a parabolic
shape of the profile is expected at the best fit
values (\todo{Reference to modeling / fitting section?}).

The result of the flux point estimation are either stored in a
\code{FluxMaps} or \code{FluxPoints} object. Both objects
are based on an internal representation of the flux which is
independent of the SED type. The flux is represented by a
the reference spectral model and an array of
normalisation values given in energy, time and spatial bins,
which factorise the deviation of the flux in the given
bin form the reference spectral model. This allows
user to conveniently transform between different
SED types. See Table \ref{tab:sed_types}.
The actual flux values are obtained by mutiplication of the norm with the
reference flux.

\begin{table*}
    \begin{center}
        \begin{tabular}{lll}
         \hline
         Type & Description & Unit Equivalency \\
         \hline
         dnde & Differential flux at a given energy & $\mathrm{TeV^{-1}~cm^{2}~s^{-1}}$ \\
         e2dnde & Differential flux at a given energy  & $\mathrm{TeV~cm^{2}~s^{-1}}$ \\
         flux & Integrated flux in agiven energy range & $\mathrm{cm^{2}~s^{-1}}$ \\
         eflux & Integrated energy flux in agiven energy range & $\mathrm{erg~cm^{2}~s^{-1}}$
        \end{tabular}
    \end{center}
    \label{tab:sed_types}
    \caption{Definition of sed types.}
\end{table*}


Both result objects support the possibility to serialise
the data into multiple formats. This includes the
GADF SED (reference?) format, FITS based ND sky-maps
and formats compatible with Astropy's \code{Table} and
\code{BinnedTimeSeries} data structures. This allows
users to directly further analyse the results, e.g.
standard algorithms for time analysis such as
computing Lomb-Scargle periodograms or Bayesian
blocks for time series. So far \gammapy does not
support "unfolding" of \gammaray spectra.
Methods for this will be implemented in future version of \gammapy.

Code example ~\ref{fig*:minted:gp_estimators} shows how to use
the \code{TSMapEstimator} objects with a given input \code{MapDataset}.
In addition to the model it allows to specify the energy
bins of the resulting flux and TS maps.


\begin{figure}
	\import{code-examples/generated/}{gp_estimators}
	\caption{Using the \code{TSMapEstimator} object from \code{gammapy.estimators} to compute a
        a flux, flux upper limits and TS map. The additional parameters \code{n\_sigma}
        and \code{n\_sigma\_ul} define the confidence levels (in multiples of the normal distribution width)
        of the flux error and and flux upper limit maps respectively.
    }
    \label{fig*:minted:gp_estimators}
\end{figure}

\subsection{gammapy.analysis}
\label{ssec:gammapy-analysis}
\todo{Quentin, Axel...}
High level analysis API
\subsection{gammapy.stats}
\label{ssec:gammapy-stats}
\todo{Regis Terrier}


The gammapy.stats subpackage contains the fit statistics and associated statistical estimators
that are commonly used in gamma-ray astronomy.
In general, gamma-ray observations count Poisson-distributed events at various sky positions,
and contain both signal and background events. Estimation of the number of signal events is done
through likelihood maximization. In Gammapy, the fit statistics are Poisson log-likelihood functions
normalized like chi-squares, i.e. they follow the expression $2 \times log L$, where $L$ is the
likelihood function used.

The statistic function used when the expected number of background events is known
is \emph{Cash}. It used When the number of background events is unknown, one has to
use a background estimate :math:`n_{bkg}` taken from an off measurement where only background events
are expected. In this case, the statistic function is ``WStat`` (see :ref:`wstat`).
\subsection{gammapy.visualization}
\label{ssec:gammapy-visualization}
\todo{Axel Donath}
The \code{gammapy.visualization} sub-package contains helper functions
for plotting and visualizing analysis results and \gammapy~data structures.
This includes for example the visualization of reflected background regions across
multiple observations or plotting large parameter correlation matrices of
\gammapy models. It also includes a helper class to split
wide field Galactic survey images across multiple panels to fit a standard
paper size.

The sub-package also provides matplotlib implementations of specific
colormaps for false color image representation. Those colormaps have
been used historically larger collaborations in the very high energy
domain (such as \milagro or \hess) as "trademark" colormaps.
While we explicitly discourage the use of those colormaps for publication
of new results, because they do not follow modern visualization
standards, such as linear brightness gradients and accessibility
for visually impaired people, we still consider the colormaps
useful for reproducibility of past results.

\subsection{gammapy.astro}
\label{ssec:gammapy-astro}
\todo{Atreyee}

The \code{gammapy.astro} sub-package contains utility functions for studying physical
scenarios in high energy astrophysics. The \code{gammapy.astro.darkmatter} module
computes the so  called J-factors and the associated gamma-ray spectra expected
from annihilation of dark matter in different channels according to the recipe
described in \cite{2011JCAP...03..051C}.

In the \code{gammapy.astro.source} sub-module, dedicated classes exist for modeling
galactic \gammaray sources according to simplified physical models, eg: SNR evolution
models \citep{1950RSPSA.201..159T, 1999ApJS..120..299T}, evolution of PWN during the
free expansion phase \citep{2006ARA&A..44...17G} or computation
of physical parameters in a pulsar assumed to be a simple dipole.

In the \code{gammapy.astro.population} sub-module there are dedicated tools
for simulating synthetic populations based on physical models derived from
observational or theoretical considerations for different classes of Galactic
VHE gamma-ray emitters: PWNe, SNRs \cite{1998ApJ...504..761C},
pulsars \cite{2006ApJ...643..332F, 2006MNRAS.372..777L, 2004A&A...422..545Y}
and gamma-ray binaries.

While the present list of use cases is rather preliminary, this can be enriched
with time with by users and/or developers according to future needs.

\subsection{gammapy.catalog}
\label{ssec:gammapy-catalog}
Comprehensive source catalogs are increasingly being provided by many high
energy astrophysics experiments. The \code{gammapy.catalog} sub-packages
provides a convenient access to the most important \gammaray catalogs.
Catalogs are represented by the \code{SourceCatalog} object, which
contains the actual catalog as an Astropy \code{Table} object.
Objects in the catalog can be accesed by row index, name of the
object or any association or alias name listed in the catalog.

Sources are represented in \gammapy by the \code{SourceCatalogObject}
class, which has the responsibility to translate the information
contained in the catalog to other \gammapy objects. This includes
the spatial and spectral model of the source, flux points and
light curves (if available) for individual objects. This
module works independently from the rest of the package, and the required
catalogs are supplied in \code{GAMMAPY\_DATA} repository.
The overview of currenly supported catalogs, the corresponding
\gammapy classes and references are shown in Table~\ref{tab:catalogs}.
Newly released relevant catalogs will be added in future.

\begin{table*}[ht!]
    \begin{center}
        \begin{tabular}{llll}
         \hline
         Class Name & Shortcut & Description & Reference\\
         \hline
         \code{SourceCatalog3FGL} & \code{"3fgl"} & 3\textsuperscript{rd} catalog of \fermi sources & \cite{3FGL} \\
         \code{SourceCatalog4FGL} & \code{"4fgl"} & 4\textsuperscript{th} catalog of \fermi  sources & \cite{4FGL} \\
         \code{SourceCatalog2FHL} & \code{"2fhl"} & 2\textsuperscript{nd} catalog high energy \fermi  sources & \cite{2FHL} \\
         \code{SourceCatalog3FHL} & \code{"3fhl"} & 3\textsuperscript{rd} catalog high energy \fermi  sources & \cite{3FHL} \\
         \code{SourceCatalog2HWC} & \code{"2hwc"} & 2\textsuperscript{nd} catalog of \hawc sources & \cite{2HWC} \\
         \code{SourceCatalog3HWC} & \code{"3hwc"} & 3\textsuperscript{rd} catalog of \hawc sources & \cite{3HWC} \\
         \code{SourceCatalogHGPS} & \code{"hgps"} & \hess Galactic Plane Survey catalog & \cite{HGPS} \\
         \code{SourceCatalogGammaCat} & \code{"gammacat"} & Open source data collection & \citep{gamma-cat} \\
         \hline
         \end{tabular}
    \end{center}
    \label{tab:catalogs}
    \caption{Overview of supported catalogs in \code{gammapy.catalog}.}
\end{table*}

\begin{figure}
	\import{code-examples/generated/}{gp_catalogs}
	\caption{Using \code{gammapy.catalogs} to access the underlying model, flux points and
		light-curve from the \fermi 4FGL catalog for the blazar PKS 2155-304}
	\label{codeexample:data}
\end{figure}

%\subsection{gammapy.utils}
\label{ssec:gammapy-utils}
Utility functions...

