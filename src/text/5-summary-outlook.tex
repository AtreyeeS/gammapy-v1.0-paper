\section{Summary and Outlook}
\label{sec:summary-and-outlook}
\todo{Axel and Regis write this...}

In this manuscript we presented the first long term support  version of \gammapy.
\gammapy is a Python package for \gammaray astronomy, which is build on the
scientific Python ecosystem, including Numpy, Scipy and Astropy as
main dependencies. It supports high-level analysis of astronomical \gammaray
data from intermediate level data formats, such as the FITS based
\gadf. Starting from lists of \gammaray events and corresponding description
of the instrument response users can reduce and project the data
to WCS, HEALPix and region based data structures. The reduced data is bundled
into datasets, which serve as a basis for Poisson maximum likelihood
modelling of the data. For this purpose \gammapy provides a wide selection
of built-in, spectral, spatial and temporal models as well as unified
fitting interface with connection to multiple optimization backends.

Around the core Python package a large diverse community of
users and contributors has developed. With regular developer meetings,
coding sprints and in-person user tutorials at relevant conferences
and collaboration meetings, the community has constantly grown.
So far \gammapy has seen O(100) contributors from ~10 different countries.
With typically ~10 regular contributors at any given time of the
project, the code base has constantly improved its range of features
and code quality. With \gammapy being selected as the base library
for the future science tools for CTA, we expect the community to grow
even further, providing a stable perspective for further usage,
development and maintenance of the project.

Besides the future use by the CTA community \gammapy has already
been used for analysis of data from the \hess and \magic instruments
With the availability of more \gammaray data from existing Cherenkov
telescopes, \gammaray

While \gammapy was mainly developed for the science community around
IACT instruments, the data model and software design is general
enough to be applied to other \gammaray instruments as well.
The use of \gammapy for the analysis of data from the High Altitude
Water Cherenkov Observatory (HAWC) has been successfully
demonstrated by \todo{Olivera et al.}. This makes \gammapy
a viable choice for the base library for the science tools
of the future Southern Widefield Gamma Ray Observatory
(SWGO) as well. \gammapy has the potential to further unify the community
of \gammaray astronomers, by sharing common tools and
a common vision of open and reproducible science for the future.
