\subsection{gammapy.maps}
\label{ssec:gammapy-maps}
\todo{Laura Olivera-Nieto}
The \verb|gammapy.maps| sub-package provides classes for representing data structures associated with a set of coordinates or a region on a sphere. In addition it allows to handle an arbitrary number of non-spatial data dimensions, such as time or energy. It is organized around three types of structures: geometries, sky-maps and map axes, which inherit from the base classes Geom, Map and MapAxis respectively.

The geometry defines the pixelization scheme and map boundaries. It also provides methods to transform between sky and pixel coordinates. Maps consist of a geometry instance together with a data array containing the corresponding map values. Map axes contain a sequence of ordered values which define bins on a given dimension, spatial or not. Map axes can have physical units attached to them, as well as non-linear step sizes.

The sub-package provides a uniform API for the FITS World Coordinate System (WCS)~\citep{Calabretta2002}, the HEALPix pixelization~\citep{Gorski2005} and region-based data structures.

% itemize because it helps me write, could also just be paragraphs...
\begin{itemize}
	\item \textbf{WCS: } The FITS WCS pixelization supports a different number of projections to represent celestial spherical coordinates in a regular rectangular grid. Gammapy provides full support to data structures using this pixelization scheme.
	\item \textbf{HEALPix: } This pixelization provides a subdivision of a sphere in which each pixel covers the same surface area as every other pixel. As a consequence, however, pixel shapes are no longer rectangular, or regular. \gammapy provides limited support to HEALPix-based maps, relying in some cases on projections to a local WCS grid.
	\item \textbf{Region geometries: } In this case, instead of a fine spatial grid dividing a rectangular sky region, the spatial dimension is reduced to a single bin with an arbitrary shape, describing a region in the sky with that same shape. Typically they are is used together with a non-spatial dimension, for example an energy axis, to represent how a quantity varies in that dimension inside the corresponding region.
\end{itemize}

Additionally, the MapAxis class provides a uniform API for axes representing bins on any physical quantity, such as energy or angular offset. The special case of time is covered by the dedicated TimeMapAxis, which allows time bins to be non-contiguous, as it is often the case with observation time-stamps. The generic class LabelMapAxis allows the creation of axes for non-numeric entries.

\begin{figure}
	\import{code-examples/generated/}{gp_maps}

	\caption{Using gammapy.maps to create a WCS, a HEALPix and a region map. Note the uniform API for map creation.}
	\label{ig*:minted:gp_maps}
\end{figure}