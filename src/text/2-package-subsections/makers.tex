\subsection{gammapy.makers}
\label{ssec:gammapy-makers}
\todo{Regis Terrier}


The data reduction contains all tasks required to process and prepare data at the DL3
level for modeling and fitting. The \texttt{gammapy.makers} sub-package contains the various
classes and functions required to do so.
First, events are binned and IRFs are interpolated and
projected onto the chosen analysis geometry. This produces counts, exposure, background, psf
and energy dispersion maps. The \texttt{MapDatasetMaker} and \texttt{SpectrumDatasetMaker}
are responsible for this task, see Fig~\ref{ig*:minted:gp_makers}.

Because the background models suffer from strong uncertainties it is required to correct
them from the data themselves. Several techniques are commonly used in gamma-ray astronomy
such as field-of-view background normalization or background measurement in reflected regions 
regions, see~\cite{Berge07}. Specific \texttt{Makers} such as the \texttt{FoVBackgroundMaker}
or the \texttt{ReflectedRegionsBackgroundMaker} are in charge of this step.

Finally, to limit other sources of systematic uncertainties, a data validity domain is
determined by the \texttt{SafeMaskMaker}. It can be used to limit the extent of the field
of view used or to limit the energy range to e.g. a domain where the energy reconstruction
bias is below a given value.



\begin{figure}
	\import{code-examples/generated/}{gp_makers}

	\caption{Using gammapy.makers to reduce DL3 level data into a \texttt{Dataset}.}
	\label{ig*:minted:gp_makers}
\end{figure}
